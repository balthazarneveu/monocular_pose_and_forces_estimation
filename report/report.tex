\documentclass[sigconf, nonacm]{acmart}
\settopmatter{printfolios=true}
\settopmatter{printacmref=false}
%% PACKAGES
\usepackage{graphicx}
\usepackage{hyperref}
\usepackage{cleveref}
\usepackage{subcaption}
\usepackage{natbib}
\usepackage{mathtools}
\usepackage{xcolor}

%% COLORS
\definecolor{darkgreen}{rgb}{0,0.8,0}

%% TITLES
\title{Review of Estimating 3D Motion and Forces of Person-Object Interactions from Monocular Video}

%% AUTHORS
\author{Balthazar Neveu}
\affiliation{%
  \institution{ENS Paris-Saclay}
  \city{Saclay}
  \country{France}
}
\email{balthazar.neveu@ens-paris-saclay.fr}

\author{Matthieu Dinot}
\affiliation{%
  \institution{Ecole Polytechnique}
  \city{Palaiseau}
  \country{France}
}
\email{matthieu.dinot@polytechnique.edu}

%% MAIN DOCUMENT
\begin{document}

  %% KEYWORDS
  \keywords{Perception and Estimation, Optimal Control, Inverse Problems}

  %% Teaser figure
  \begin{teaserfigure}
    \includegraphics[width=0.6\textwidth]{figures/authors_original_pipeline.png}
    \centering
    \caption{Full original pipeline. 
    A multiple stage vision pipeline extracts poses and contact data of the human and the object.
    Full body dynamics (joint angles, torques and external forces) are then reconstructed during the optimization stage.
    }
    \label{fig:original_pipeline}
  \end{teaserfigure}

  %% TITLE
  \maketitle




  %% CONTENT
  \section{Introduction}
\label{sec:intro}

Humans can easily learn how to perform a given movement or manipulate an object by observing others and attempting to replicate their actions. 
In robotics, this behavioral cloning ability would be usefull to train humanoid robots to perform such tasks. 
However, unlike humans, computers still struggle to understand complex human-object interactions from visual data alone. 
Traditional methods like motion capture offer a solution, but they come with high costs and do not leverage the abundant instructional video 
resources available online.

To address this challenge, \citet{li2019estimating} proposed a novel approach for estimating 3D motion and forces of human-object interactions 
from single RGB videos. 

In this work, we critically reviewed and reimplemented certain aspects of their method. Our initial goal was to assess 
the reproducibility and robustness of the original findings, as well as to explore potential enhancements. We focused on understanding the 
underlying methodology, adapting it to a simplified pipeline. Our code is available on~\href{https://github.com/balthazarneveu/monocular_pose_and_forces_estimation}{GitHub}.



  \section{Methodology of the Original Paper}
\label{sec:methodo_paper}

The study by \citet{li2019estimating} aimed to reconstruct the 3D motion of a person interacting with an object, including the position 
and applied forces, from a single RGB video. Their methodology unfolds in two principal steps. The first step involves the application of 
computer vision techniques to accurately determine the human and object poses, as well as the points of contact between them and with the 
surrounding environment. The second step solves an optimization problem, rooted in control theory principles, to fully recover 
the dynamic motion.  


\subsection{Retrieve Human and Object 3D Pose and Contact from RGB Video}
\label{subsec:retrieve_original}

\noindent\textbf{Human 2D Pose Estimation.}\label{2dpose} For the task of human pose estimation, the authors utilized a pretrained version of 
OpenPose~\cite{cao2017realtime}. This tool enabled them to identify and track human joint positions in two dimensions from the RGB video.

\noindent\textbf{Contact Recognition.} The recognition of contact points, being whether specific joints (like feet, hands, and neck) were in 
contact with the ground or the object, was achieved by first cropping areas in the video frame around the joints identified by OpenPose. 
These cropped images were then processed through the corresponding joint's ResNet \cite{he2016deep}, which had been trained on a custom dataset 
to discern contact.

\noindent\textbf{Object 2D Endpoint Detection.} For estimating the 2D positions of object endpoints, the authors utilized instance segmentation 
capabilities of Mask R-CNN \cite{he2017mask}. This method was applied to different classes of objects (such as barbells, hammers, scythes, 
and spades) identified in their video dataset. Mask R-CNN, trained separately for each object class, enabled the extraction of segmentation masks
and bounding boxes from the video frames. These were then used to estimate the 2D locations of the object's endpoints.

\noindent\textbf{Human 3D Pose Estimation.} The authors also utilized HMR~\cite{kanazawa2018end} for 3D human pose estimation, although this 
is not clearly stated in their method overview.


\subsection{Reconstruct Person-Object Dynamic Motion}
\label{subsec:reconstruct_original}

The optimization framework presented by \citet{li2019estimating} is designed to estimate the 3D trajectories of both the human and the object, 
as well as the exerted contact forces, by solving the following trajectory optimization problem: 

\begin{equation*}
    \begin{aligned}
        & \underset{x,u,c}{\text{minimize}}
        & & \int_{0}^{T} l^h(x, u, c) + l^o(x, u, c)dt, \\
        & \text{subject to}
        & & \kappa(x, c) = 0 \quad (\text{contact motion model}), \\
        &&& \dot{x} = f (x, c, u) \quad (\text{full-body dynamics}), \\
        &&& u \in \mathcal{U} \quad (\text{force model}).
    \end{aligned}
\end{equation*}

\noindent\textbf{Variables.} The state variables \(x \coloneqq (q^h, q^o, \dot{q}^h, \dot{q}^o)\) encompass the configuration and velocities of 
the human and object, while the control variables \(u \coloneqq (\tau_m^h, f_k, k=1, \dots, K)\) include the muscle torques and contact forces 
at the \(K\) contact points. The contact state \(c\) represents the presence and locations of contacts throughout the video sequence. 

It should be noted, although not explicitly stated in the paper, that in their implementation, the velocities (\(\dot{q}^h\) and \(\dot{q}^o\))
are not directly used as variables. Instead, velocities are approximated using the backward finite difference scheme 
\(\dot{q}_t = (q_t - q_{t - 1}) / \Delta t\).

\noindent\textbf{Loss Functions.} The loss functions \(l^h\) and \(l^o\) for the human and object respectively consist of the following components:

\begin{itemize}
    \item 
        \(l_{2D}\): A term ensuring 2D consistency by minimizing the error between the observed 2D joint positions given by OpenPose and 
        their re-projections from the 3D estimates, using the camera projection matrix \(P_{\text{cam}}\). Outlier influence is mitigated 
        by employing the Huber loss function \(\rho\).
    \item 
        \(l_{3D}\): This loss ensuring that the estimated 3D positions adhere closely to the HMR references.
    \item 
        \(l_{\text{pose}}^h\): A likelihood term for the human pose, encouraging plausible human poses through a Gaussian Mixture Model 
        trained on motion capture data.
    \item
        \(l_{\text{torque}}^h\): A regularization term penalizing high muscle torque values to favor energy-efficient movements.
    \item 
        Motion smoothness: Terms penalizing rapid movements or accelerations to align with the typically smooth human motion observed over 
        the video frame rate.
    \item
        Force-related terms: These encompass additional forces and contact forces that were not the focus of further investigation in our 
        study.
\end{itemize}

The loss function for the object, \(l^o\), is constructed similarly, omitting the 3D consistency and pose likelihood terms due to the nature 
of object motion and the absence of a pose prior.
  \section{Reimplementation Strategy}
\label{sec:remplementation}

The code accompanying the original paper  is poorly written, poorly documented, and is formed of several complex and independant blocks. 
To effectively explore the core contributions of the paper, we decided to reimplement it with a more focused scope.

In our approach, we focused on reconstructing arm motion from video data, a simplification that allowed us to tackle the fundamental 
aspects of the methodology without the computational burden of full-body modeling.

\subsection{Vision Pipeline}
\label{subsec:vision_pipeline}

For 2D and 3D human pose estimation, we utilized the MediaPipe framework~\cite{lugaresi2019mediapipe}, which offers a more modern alternative 
to OpenPose. MediaPipe simplifies the vision pipeline by also replacing HMR in our usecase. An illustration of MediaPipe's inference results 
can be seen in~\cref{fig:mediapipe}. We then only used the arm joint 2D and 3D positions.

\begin{figure}
    \centering
    \includegraphics[width=8cm]{figures/pose_detection_mediapipe_collage.png}
    \caption{From the full body pose estimated with MediaPipe ~\cite{lugaresi2019mediapipe}, 
    we extract the arm joints for left and right arms. As a matter of homogeneous graphics,
    shoulder will always appear as red, elbow as green and wrist as blue. Upper arm will appear as green and forearm as blue.}
    \label{fig:mediapipe}
\end{figure}

Although we experimented with the contact recognizer component, we did not use it into our pipeline since we do not model contact.

\subsection{Arm Model}

We modelled the arm in Pinocchio~\cite{carpentier2019pinocchio}. The shoulder is modelized as a spherical joint, and the elbow is modelized 
as a revolute joint.

\subsection{Inverse Kinematics}
To get proper initializations to feed to the inverse dynamics optimizer, we relied on the 3D positions of the arm joints estimated by MediaPipe.
From a sequence of 3D positions of the arm joints, we used the inverse kinematics approach to estimate the joint angles.
Inverse kinematics is higly relying on the Pinocchio library ~\cite{carpentier2019pinocchio} to compute 
the Jacobian and perform forward kinematics (compute the 3D positions  $M(t)_{j}$ of the joints from the joint angles $q(t)$).
% @TODO: Add equations for IK with the 2 tasks (elbow and wrist 3D positions)

Devil is in the details, and we had to be careful about the following points:
\begin{itemize}
    \item We provide 2 tasks to the inverse kinematics solver: one for the elbow 3D position and one for the wrist 3D position. 
    Order matters, elbow comes first.
    \item The mediapipe 3D estimations are not perfect but the most nottable issue is that 
    the upper arm and the forearm lengths are not constant across time. To overcome this limitation,
    3D positions are re-scaled to match the expected nominal model arm length. 
    This arm length normalization ensures that a solution exists but on the other hand, it may introduce reprojection errors.
\end{itemize}
Although nothing grants temporal smoothness of the estimated joint angles, we recursively feed the previous solution as an initial guess to the current timestep.
This simple trick gives good results in practice (assessed qualitatively on the match between the moving arm and the video).


\subsection{Camera position}
\subsubsection{Problem formulation}
Since the upper arm can rotate in all directions and since we're not using any priors to constrain the arm motion (at the shoulder level),
moving the camera around the arm is equivalent to spinning the arm around itself. 
Our simplification unfortunately leads to a degenerate problem where the camera rotation may end up being redundant with the arm rotation.
To avoid this issue, we decided to fix the camera orientation. 
We estimate the camera 3D position $P_{\text{cam}}(t)$ 
by jointly minimizing the 2D reprojection error of joints.
$$\underset{P_{\text{cam}}(t)}{min}\sum_{j\in{\text{shoulder, elbow, wrist}}} ||K.\big[Q_{\text{cam}} | P_{\text{cam}}(t)] \vec{X}^{(j)}(t) - \vec{x}^{(j)}(t) ||$$
where
\begin{itemize}
    \item $K$ is the camera 3x3 intrinsic matrix
    \item $Q$ is the camera orientation set to identity here $I_{3}$
    \item $-P_{\text{cam}}(t)$ is the 3D camera position in the world frame (meters)
    \item $\vec{X}^{(j)}$ are the homogeneous 3D coordinates of joint $j$ (meters)
    \item $\vec{x}^{(j)}$ are the homogeneous 2D coordinates of joint $j$ in the sensor frame (pixels).
\end{itemize}

\subsubsection{Camera calibration}
To reduce the number of degrees of freedom, we always used the same camera through all our tests and 
we pre-calibrated the camera intrinsics $K$ using a 7x10 printed checkerboard 
and the OpenCV implementation of the Zhang's method~\cite{Zhang00calib}.
Below, we detail the focal length estimation from camera specifications.
Xiaomi Mi11 Ultra main camera ($2.8\mu m$ pixel pitch) specifications in photo mode:
\begin{itemize}
    \item 24mm focal length - full frame (24x36mm) equivalent
    \item Sensor size 4000x3000 = 12Mpix
\end{itemize}
We end up with a focal length for the photo mode of $f_{\text{pix}}^{\text{photo}}  = 24mm * 4000px / 36mm = 2666px$.

But since we're using a FullHD video mode with a crop factor of around 15\% on each side,
it is needed to rescale the focal length acordingly $f_{\text{pix}}^{\text{video}} = 2666px * 1.3 * \frac{1920px}{4000px} \approx 1664px$.
Calibration method provides a estimated focal length of $1690px$ which is close enough to the specifications.
We assume the camera to be a pinhole and neglect radial distortion.

\subsubsection{Camera pose optimization}
Each timestep can be viewed as a separate optimization problem. We can also ensure better temporal coherence
by adding a regularization term on the camera 3D position to the previous cost function $||\frac{\Delta P_{\text{cam}}}{\Delta{t}}||^2$.
We perform a first quick pass using small windows of 2 second (60 frames) to initialize the camera position sequence.
In a second pass, we optimize over the whole sequence at once.
A rule of thumb on a i7 - 8 core CPU is that this optimizer takes roughly 1 minute to optimize the camera pose a 1 minute long video.

\begin{figure}
    \centering
    \includegraphics[width=9cm]{figures/camera_pose_fitting_collage.png}
    \caption{Camera pose fitting on the first 100 frames of the video (doing a circle with the arm). Average reprojection error reported here are computed on the 3 arm joints over the whole video.
    Estimated distance from the camera to the shoulder is 3 meters with is faithful to the real distance.}
    \label{fig:camera_fitting}
\end{figure}

\subsection{Optimizer}


  \section{Discussion on the Paper}
\label{sec:discussion}

Our review and partial reimplementation of the work by~\citet{li2019estimating} have led us to several observations about 
the original study.

\begin{itemize}
    \item \textbf{Method and Code Complexity:} The complexity of the approach, both in terms of the underlying methodology and the associated 
    codebase, poses significant barriers to replication and extension.

    \item \textbf{Challenging Loss Coefficient Tuning:} Fine-tuning the loss coefficients in the optimization problem presents a significant 
    challenge. The balance between data fidelity and regularization, especially concerning torques, requires careful calibration, which may 
    limit the method's generalizability to diverse datasets. 

    \item \textbf{Confidence in Force and Torque Estimations:} Assessing the accuracy of reconstructed forces and torques is problematic due 
    to the limited benchmarks available for validation.

    \item \textbf{Consistency in Object Weight and Body Dimension Priors:} In their method, weights and weight matricies are predefined. 
    But the movement heavily depends on the weight and dimensions of the manipulated object and the human body, which can vary significantly 
    across different videos. 

    \item \textbf{Relevance of Full Body Dynamics:} Given the necessary approximations in velocity and acceleration, and the inability to 
    strictly enforce full-body dynamics, the added value of attempting to reconstruct these dynamics as opposed to focusing 
    solely on kinematics can be questionned. A comparative analysis with kinematics-focused methods could be insightful.

    \item \textbf{Lack of Data Adimensionality in Optimization:} The absence of adimensionalization or standardization in data processing could 
    affect the model's ability to generalize. This issue becomes particularly evident in scenarios where similar movements are performed at 
    different speeds, potentially leading to varied results.

\end{itemize}
  \section{Conclusion}
\label{sec:conclusion}

In conclusion, our review and reimplementation of~\citet{li2019estimating}'s paper provided valuable insights into the complexities 
and challenges of estimating 3D motion and forces in human-object interactions from RGB videos. 
The original work stands as a significant novel contribution to this field. While our reimplementation is simplified and restricted, 
and faced difficulties, particularly in applying the method to real-world data, it still highlighted some limitations of the original method.



  \newpage

  %% BIBLIOGRAPHY
  \bibliographystyle{ACM-Reference-Format}
  \bibliography{references}


  %% APPENDIX
  \appendix
  \section{Appendix}
\subsection{Camera calibration}
\label{app:cam_calib}
Since we're working in a controlled environment (not in the wild, unlike the original paper), we calibrate a single camera once and for all,
\begin{itemize}
    \item using a 7x10 printed checkerboard shot in various orientations
    \item  using the OpenCV implementation of the Zhang's method~\cite{Zhang00calib}.
\end{itemize}

Below, we detail the focal length estimation from camera specifications
and make sure it matches with the calibration estimation.
Xiaomi Mi11 Ultra main camera ($2.8\mu m$ pixel pitch) specifications in photo mode:
\begin{itemize}
    \item 24mm focal length - full frame (24x36mm) equivalent
    \item Sensor size 4000x3000 = 12Mpix
\end{itemize}
We end up with a focal length for the photo mode of $f_{\text{pix}}^{\text{photo}}  = 24mm * 4000px / 36mm = 2666px$.

But since we're using a FullHD video mode with a crop factor of around 15\% on each side,
it is needed to rescale the focal length acordingly $f_{\text{pix}}^{\text{video}} = 2666px * 1.3 * \frac{1920px}{4000px} \approx 1664px$.
Calibration method provides a estimated focal length of $1690px$ which is close enough to the specifications.
We assume the camera to be a pinhole and neglect radial distortion.


\subsection{Code description}
\label{app:code}
The code is available at:

~\href{https://github.com/balthazarneveu/monocular_pose_and_forces_estimation}{github.com/balthazarneveu/monocular\_pose\_and\_forces\_estimation}

The code is written in Python and relies on a several external libraries:
\begin{itemize}
    \item Pinocchio for kinematics and dynamics computations aswell as the arm model.
    \item Meshcat for 3D visualization.
    \item OpenCV for the camera calibration and the image processing.
    \item MoviePy wraps video processing.
    \item ~\href{https://developers.google.com/mediapipe}{Google Mediapipe} for 2D and  3D pose estimation.
    \item Scipy for the Levenberg-Marquardt optimization.
    \item ~\href{https://github.com/emmcb/batch-processing}{batch-processing} to process multiple video files in a systematic way.
    \item ~\href{https://github.com/balthazarneveu/interactive_pipe}{interactive-pipe} to display a GUI with graphs and images and interact with sliders and keyboard.
    This library works with Matplotlib as the default graphical backend but PyQT/PySide is highly recommended for the demo.
\end{itemize}

To process a new set of videos, located in the \texttt{data} folder, run the following command:
\begin{verbatim}
    python scripts/batch_video_processing.py
    -i "data/*.mp4"
    -o "out"
    -A demo
\end{verbatim}
\textit{When the GUI pops up, press F1 to get the help menu to learn about the shortcuts. Press F11 to display in full screen.
Do not forget to click the hyperlink in the terminal to open the MeshCat viewer in your browser.}


If you're using a different camera, you'll need to calibrate your camera intrinsics first.
Capture a calibration video sequence using a 10x7 checkerboard (print it and stick it on a cardboard or display it on your screen).

\begin{verbatim}
    python scripts/batch_video_processing.py
    -i "data/camera_calibration_<cam_id>.mp4"
    -o "calibration"
    -A camera_calibration
\end{verbatim}

Then you'll be able to process your pose videos using the new camera intrinsics, by simply specifiying
the calibration file path:
\begin{verbatim}
    -calib "calibration/camera_calibration_<cam_id>.yaml"
\end{verbatim}

The core of the code for inverse kinematics and inverse dynamics is located in:
~\href{https://github.com/balthazarneveu/monocular_pose_and_forces_estimation/tree/main/src/projectyl/dynamics}{src/projectyl/dynamics}
  

\end{document}
