\section{Reimplementation Strategy}
\label{sec:remplementation}

The code accompanying the original paper  is poorly written, poorly documented, and is formed of several complex and independant blocks. 
To effectively explore the core contributions of the paper, we decided to reimplement it with a more focused scope.

In our approach, we concentrated on reconstructing arm motion from video data, a simplification that allowed us to tackle the fundamental 
aspects of the methodology without the computational overhead of full-body modeling.

\subsection{Vision Pipeline}
\label{subsec:vision_pipeline}

For 2D and 3D human pose estimation, we utilized the MediaPipe framework~\cite{lugaresi2019mediapipe}, which offers a more modern alternative 
to OpenPose. MediaPipe simplifies the vision pipeline by also replacing HMR in our usecase. An illustration of MediaPipe's inference results 
can be seen in~\cref{fig:mediapipe}.

To recover the 2D and 3D human pose, we relied on the MediaPipe framework~\cite{lugaresi2019mediapipe}. It is more modern than OpenPose, and 
simplify the vision pipeline, as we don't need HMR anymore. See a visualization of MediaPipe inference on~\cref{fig:mediapipe}.

\begin{figure}
    \centering
    \fbox{\rule{0pt}{2in} \rule{0.9\linewidth}{0pt}}
    \caption{Beautiful picture of Balthazar with MediaPipe.}
    \label{fig:mediapipe}
\end{figure}

We then only used the arm joint 2D and 3D positions.

Although we experimented with the contact recognizer component, we did not use it into our pipeline since we do not model contact.

\subsection{Arm Model}

We modelized the arm in Pinocchio~\cite{carpentier2019pinocchio}.

\subsection{Inverse Kinematics}

\subsection{Optimizer}

